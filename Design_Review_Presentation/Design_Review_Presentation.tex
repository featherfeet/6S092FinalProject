\documentclass{beamer}

\usepackage{graphicx}
\usepackage{listings}

\title{6.S092 Track 2 Project: Amateur Radio Satellite Phased Array Receiver}
\author{Oliver Trevor}
\institute{MIT}
\date{January 25, 2023}

\begin{document}
    \frame{\titlepage}

    \begin{frame}
        \frametitle{Background}

        \begin{itemize}
            \item There are various satellites in orbit with amateur radio payloads, usually repeaters and sometimes digital transponders

            \item Licensed amateurs can use these for communications or experimentation, which is cool because it's space

            \item Since the duplexer (cavity filters) that ground-based repeaters carry is too large for a CubeSat, these satellites usually uplink on VHF and downlink on UHF (sometimes this is reversed)

            \item Ham radio operators usually use motorized mounts or manual, by-hand tracking to aim an antenna at a satellite to pick up its weak downlink signal

            \item Receiving signals is possible with even a handheld radio, but the antenna must be very directional

            \item Modern electronically-steered antenna systems work by changing the phase relationships between RF signals to create constructive interference in a desired direction and destructive interference in other directions
        \end{itemize}

    \end{frame}

    \begin{frame}
        \frametitle{Problem Overview}
        \begin{itemize}
            \item Amateur radio satellites are a long way away and have very low-power transmitters

            \item Existing solutions for communicating with ham radio satellites usually involve mechanically-steered antennae, which are prone to failure and have limited tracking speed (and no ability to ``hunt" for a signal by sweeping the approximate region of sky)

            \item Even modern ham radio mostly does not use phased arrays, except in either fixed (not steerable) setups like the Walker roof EME array or in very simplistic designs like ``HF noise cancellers"

            \item I would like to build an electronically-steered phased array for receiving from ham radio satellites on UHF using an ordinary rig (no software-defined radios)
        \end{itemize}
    \end{frame}

    \begin{frame}
        \frametitle{Basic RF Topology}
        \begin{itemize}
            \item Four Yagi antennas, probably home-made, arranged in a square and tuned for UHF (not part of PCB)

            \item Four coax cables (not part of PCB)

            \item Four edge-mount through-hole (for strength) female SMA connectors

            \item (Possibly) four filters for the ham UHF band (420-450 MHz)

            \item Four LNAs with bias tees for power

            \item Four voltage-controlled phase-shifters

            \item RF power combiner

            \item (Possibly) RF attenuator

            \item Edge-mount through-hole female SMA connector for cable to radio
        \end{itemize}
    \end{frame}

    \begin{frame}
        \frametitle{Non-RF Things}
        \begin{itemize}
            \item Power circuitry to convert 13.8 V (standard for ham radio) down to 5 V for the LNAs, possibly also to a stable reference voltage for the DACs

            \item Microcontroller for electronic phase steering

            \item Four DACs for phase control signals

            \item Four op-amps for phase control signals
        \end{itemize}
    \end{frame}

    \begin{frame}
        \frametitle{Expected Difficulties}

        \begin{itemize}
            \item Literally everything because this is RF

            \item Impedance-matching traces

            \item Preventing MCU and power circuitry from coupling interference into RF things

            \item Preventing nearby ham transmitter from crashing/damaging MCU with induced fields

            \item Routing control and power traces without interrupting RF traces
        \end{itemize}
    \end{frame}

    \begin{frame}
        \frametitle{Questions}

        \begin{itemize}
            \item Choice of capacitor and inductor values for LNA bias-tee circuit

            \item Do I need filtering on the input for the LNAs to work properly, or can I just let the radio at the end do the filtering? If I need filtering, should it be implemented as discrete components on the PCB or as a module?

            \item How do I down-regulate 13.8 VDC to 5 VDC without creating RF noise? / Is the RF noise from a switching converter a problem?

            \item Some footprints for Mini-circuits modules have lots of vias in their grounds--why? Should I use those?

            \item Do I need an attenuator at the end to reduce the combined power to avoid damaging the radio?
        \end{itemize}
    \end{frame}

    \begin{frame}
        \frametitle{Questions}

        \begin{itemize}
            \item What microcontroller should I use (this choice may be dictated more by supply than by choice...)? Should the DACs be integrated with it or separate?

            \item How do I make sure the DACs output voltage will be stable/reliable when the power voltage is not (i. e. what voltage reference should I use)?

            \item Do I need RF shielding boxes around things?

            \item Should the DAC outputs / bias voltages have some kind of bypass capacitance on them to stop RF affecting them?

            \item Is it a good idea to have a USB-serial converter for talking to the MCU, or is that going to be unavoidably noisy?
        \end{itemize}
    \end{frame}

\end{document}
